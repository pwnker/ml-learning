\documentclass{article}

\usepackage{mathtools}
\usepackage{tikz}

\begin{document}
\section{Gradient Descent}

\begin{align}
dC(\omega) = \lim_{\varepsilon\to 0}\frac{C(\omega + \varepsilon) - C(\omega)}{\varepsilon} \\
\end{align}

\subsection{Derivitive Of One Parameter Function}
Within the \textit{Twice} example we described a model with one parameter - w 

\def\avgsum[#1, #2]{\frac{1}{#2}\sum_{#1=1}^{#2}}
\begin{align}
    \intertext{The formula had a form like this:}
    f(x) = x \cdot w
\end{align}
\begin{align}
    \intertext{Function $C$ which takes one parameter $w$ is defined as:}
    C(w)  = \avgsum[i, n](x_i \cdot w - y_i)^2 
\end{align}

\begin{align}
    \intertext{Let's compute the derivitive $C'$ of our function:} 
    C'(w) &= \left(C\right)' \\
          &= \left(\avgsum[i, n](x_i \cdot w - y_i)^2\right)' = \\
          &= \left(\avgsum[i, n](x_i \cdot w - y_i)^2 \right)' = \\ 
          &= \frac{1}{n} \left(\sum_{i=1}^{n}(x_i \cdot w - y_i)^2 \right)' = \\ 
          &= \avgsum[i, n] \left( (x_i \cdot w - y_i)^2 \right)' = \\ 
          &= \avgsum[i, n] \left( 2 \cdot (x_i \cdot w - y_i) (x_i \cdot w - y_i)' \right) = \\ 
          &= \avgsum[i, n] \left( 2 \cdot (x_i \cdot w - y_i) \cdot x_i \right) 
\end{align}


\begin{align}
    \intertext{The final form of our derivitive:}
    C'(w) &= \avgsum[i, n] \left( 2 \cdot (x_i \cdot w - y_i) \cdot x_i \right) 
\end{align}

\subsection{One Neuron Model With 2 Inputs}
One neuron model is defined as:  
\begin{align}
    z  = \sigma(x \cdot w_1 + y \cdot w_2 + b) 
\end{align}
\begin{align*}
    \text{$x_1$}  \: &\text{...} \: \text{input parameter}\\
    \text{$x_2$}  \: &\text{...} \: \text{input parameter}\\
    \text{$w_1$}  \: &\text{...} \: \text{weight paramter}\\
    \text{$w_2$}  \: &\text{...} \: \text{weight paramter}\\
    \text{b}  \: &\text{...} \: \text{bias parameter}\\
    \sigma    \: &\text{...} \: \text{sigmoid activation function}
\end{align*}
\begin{center}
\def\d{2}
\begin{tikzpicture}
    \node (X_1) at (-\d,0.75) {$x$}; 
    \node (X_2) at (-\d,-0.75) {$y$}; 
    \node[shape=circle,draw=black] (N) at (0,0) {$\sigma, b$}; 
    \node (Y) at (\d,0) {$z$}; 
    \path[->] (X_1) edge node[above] {$w_1$} (N);
    \path[->] (X_2) edge node[above] {$w_2$} (N);
    \path[->] (N) edge (Y);
\end{tikzpicture}
\end{center}
\subsubsection{Cost}
Let's recall the Sigmoid activation function  
\begin{align}
    \sigma(x)  &= \frac{1}{1 + e^{-x}} \\ 
    \sigma(x)' &= \sigma(x) \cdot (1 - \sigma(x)) 
\end{align}
Let's define the cost function $C$ for our model
\begin{align}
    a_i  &= \sigma(x_i \cdot w_1 + y_i \cdot w_2 + b) \\ 
    C(x) &= \avgsum[i, n](a_i - y_i)^2 
\end{align}
Let's compute the derivitive $C'$ for our function\\
\\
We have to modify TWO parameters $w, b$\\
\\
For this we will use PARTIAL DERIVITIVES this means that first we compute a derivitive in respect to $w_1$, $w_2$ and then we compute another derivitive in respect to $b$ \\
\def\pd[#1]{\partial_{#1}}
\begin{enumerate}
    \item{
        Partial Derivitive in respect to $w_1$
        \begin{align}
            a_i  &= \sigma(x_i \cdot w_1 + y_i \cdot w_2 + b) = \\ 
            \pd[w_1]a_i  &= \pd[w_1](\sigma(x_i \cdot w_1 + y_i \cdot w_2 + b) = \\ 
                         &= a_i(1 - a_i) \pd[w_1](x_i \cdot w_1 + y_i \cdot w_2 +b) = \\ 
            \pd[w_1]a_i  &= a_i(1 - a_i) \cdot x_i \\ 
            \\
            \pd[w_1]C &=\pd[w_1]\left(\avgsum[i, n](a_i - z_i)^2\right) =  \\ 
                    &=\avgsum[i, n]\pd[w_1]\left((a_i - z_i)^2\right) =  \\ 
                    &=\avgsum[i, n]2 \cdot (a_i - z_i)\pd[w_1]\left(a_i - z_i\right) = \\ 
                    &=\avgsum[i, n]2 \cdot (a_i - z_i)\pd[w_1]a_i = \\ 
            \pd[w_1]C &=\avgsum[i, n]2 \cdot (a_i - z_i) \cdot a_i(1 - a_i) \cdot x_i \\ 
        \end{align}
    }
    \item{
        Partial Derivitive in respect to $w_2$
        \begin{align}
            a_i  &= \sigma(x_i \cdot w_1 + y_i \cdot w_2 + b) = \\ 
            \pd[w_2]a_i  &= \pd[w_2](\sigma(x_i \cdot w_1 + y_i \cdot w_2 + b) = \\ 
                         &= a_i(1 - a_i) \pd[w_2](x_i \cdot w_1 + y_i \cdot w_2 +b) = \\ 
            \pd[w_2]a_i  &= a_i(1 - a_i) \cdot y_i \\ 
            \\
            \pd[w_2]C &=\pd[w_2]\left(\avgsum[i, n](a_i - z_i)^2\right) =  \\ 
                    &=\avgsum[i, n]\pd[w_2]\left((a_i - z_i)^2\right) =  \\ 
                    &=\avgsum[i, n]2 \cdot (a_i - z_i)\pd[w_2]\left(a_i - z_i\right) = \\ 
                    &=\avgsum[i, n]2 \cdot (a_i - z_i)\pd[w_2]a_i = \\ 
            \pd[w_2]C &=\avgsum[i, n]2 \cdot (a_i - z_i) \cdot a_i(1 - a_i) \cdot y_i \\ 
        \end{align}
    }
    \item{
        Partial Derivitive in respect to $b$
        \begin{align}
            a_i  &= \sigma(x_i \cdot w_1 + y_i \cdot w_2 + b)  \\ 
            \pd[b]a_i &= \pd[b](\sigma(x_i \cdot w_1 + y_i \cdot w_2 + b) = \\ 
                      &= a_i(1 - a_i)\pd[b](x_i \cdot w_1 + y_i \cdot w_2 +b) = \\ 
                      &= a_i(1 - a_i) \cdot 1 = \\ 
            \pd[b]a_i &= a_i(1 - a_i) \\ 
            \\
            \pd[b]C &=\pd[b]\left(\avgsum[i, n](a_i - z_i)^2\right) =  \\ 
                    &=\avgsum[i, n]\pd[b]\left((a_i - z_i)^2\right) =  \\ 
                    &=\avgsum[i, n]2 \cdot (a_i - z_i)\pd[b]\left(a_i - z_i\right) = \\ 
                    &=\avgsum[i, n]2 \cdot (a_i - z_i)\pd[b]a_i = \\ 
            \pd[b]C &=\avgsum[i, n]2 \cdot (a_i - z_i) \cdot a_i(1 - a_i) \\ 
        \end{align}
    }
\end{enumerate}
To summarize the partial derivitives are:
\begin{align}
    a_i  &= \sigma(x_i \cdot w_1 + y_i \cdot w_2 + b) \\ 
    \pd[w_1]C &=\avgsum[i, n]2 \cdot (a_i - z_i) \cdot a_i(1 - a_i) \cdot x_i \\ 
    \pd[w_2]C &=\avgsum[i, n]2 \cdot (a_i - z_i) \cdot a_i(1 - a_i) \cdot y_i \\ 
    \pd[b]C   &=\avgsum[i, n]2 \cdot (a_i - z_i) \cdot a_i(1 - a_i) \\ 
\end{align}
\newpage
\subsection{Execution Time Comparison}
Let's compare computation time differance between \textbf{Finite Difference} and \textbf{Gradient Descent}\\
My machine is Lenovo Legion Slim 5:
\begin{itemize}
    \item {All computations are run on the CPU}
    \item {CPU: AMD Rayzen 7 7840HS (16) 5.137Ghz}
\end{itemize}
The test:
\begin{itemize}
    \item {Neural network will try to learn the proper configuration for simulating NAND gate} 
    \item {Comparison of training the model using the \textit{Finite Difference} method and \textit{Gradient Descent}} 
    \item {8.000.000 iterations(epochs) of training will be run (overkill I know)} 
\end{itemize}
\textbf{RESULTS:}\\
\indent \[ \textbf{Finite Difference}:\ \approx 1.556\ seconds \]
\indent \[ \textbf{Gradient Descent}:\ \approx 0.473\ seconds \]
\end{document}

